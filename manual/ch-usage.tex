\chapter{Using the INET Framework}
\label{cha:usage}

\section{Installation}

To install the INET Framework, download the most recent INET Framework source
release from the download link on the http://inet.omnetpp.org web site,
unpack it, and follow the instructions in the INSTALL file in the root
directory of the archive.

If you plan to simulate ad-hoc networks or you need more wireless link layer
protocols than provided in INET, download and install the INETMANET source
archive instead. (INETMANET aready contains a copy of the INET Framework.)

If you plan to make use of other INET extensions (e.g. HttpTools, VoipTools or TraCI),
follow the installation instructions provided with them. If there are no
install instructions, check if the archive contains a \ttt{.project} file.
If it does, then the project can be imported into the IDE (use File > Import >
General > Existing Project into workspace); make sure that the project is recognized
as an OMNeT++ project (the Project Properties dialog contains an OMNeT++ page)
and it lists the INET or INETMANET project as dependency (check the Project References
page in the Project Properties dialog).

If the extension project contains no \ttt{.project} file, create an empty OMNeT++
project using the New OMNeT++ Project wizard in File > New, add the INET or INETMANET
project as dependency using the Project References page in the Project Properties dialog,
and copy the source files into the project.

After installation, run the example simulations to make sure everything works
correctly. The INSTALL file also describes how to do that.


\section{INET as an OMNeT++-based simulation framework}

TODO what is documented where in the OMNeT++ manual (chapter, section title)

TODO the pattern syntax: *, **; there is also syntax to match index ranges and
ranges of numbers embedded in names


The INET Framework builds upon OMNeT++, and uses the same concept:
\tbf{modules} that communicate by message passing. Hosts, routers, switches
and other network devices are represented by OMNeT++ compound modules.
These compound modules are assembled from simple modules that represent
protocols, applications, and other functional units. A network is again an
OMNeT++ compound module that contains host, router and other modules. The
external interfaces of modules are described in NED files. NED files
describe the parameters and gates (i.e. ports or connectors) of modules,
and also the submodules and connections (i.e. netlist) of compound modules.

The internal structure of compound modules can be customized in several
ways. The first way is the use of \tbf{gate vectors and submodule vectors}.
The sizes of vectors may come from parameters or derived by the number of
external connections to the module. For example, one can have an Ethernet
switch model that has as many ports as needed, i.e. equal to the number of
Ethernet devices connected to it. The second way of customization is
\tbf{parametric types}, that is, the type of a submodule may be specified
as a string parameter. For example, the relay unit inside an Ethernet
switch has several alternative implementations, each one being a distinct
module type. The switch model contains a parameter which allows the user to
select the appropriate relay unit implementation. A third way of
customizing modules is \tbf{inheritance}: a derived module may add new
parameters, gates, submodules or connections, and may set inherited
unassigned parameters to specific values.

Modules are organized into hierarchical \tbf{packages} that directly map to
a folder tree, very much like packages in the Java language. Packages in
INET are organized roughly according to OSI layers; the top packages
include \ttt{inet.applications}, \ttt{inet.transport},
\ttt{inet.networklayer}, and \ttt{inet.linklayer}. Other packages are
\ttt{inet.base}, \ttt{inet.util}, \ttt{inet.world}, \ttt{inet.mobility} and
\ttt{inet.nodes}. These packages correspond to the \ttt{src/applications/},
\ttt{src/transport/}, etc. directories in the INET source tree. (The
\ttt{src/} directory corresponds to the \ttt{inet} package, as defined by
the \ttt{src/package.ned} file.) Subdirectories within the top packages
usually correspond to concrete protocols or protocol families. The
implementations of simple modules are C++ classes with the same name, with
the source files placed in the same directory as the NED file.

The \ttt{inet.nodes} package contains various pre-assembled host, router,
switch, access point, and other modules, for example
\nedtype{StandardHost}, \nedtype{Router} and \nedtype{EtherSwitch} and
\nedtype{WirelessAP}. These compound modules contain some customization
options via parametric submodule types, but they are not meant to be
universal: it is expected that you will create your own node models for
your particular simulation scenarios.

Network interfaces (Ethernet, IEEE 802.11, etc) are usually compound modules
themselves, and are being composed of a queue, a MAC, and possibly other
simple modules. See \nedtype{EthernetInterface} as an example.

Not all modules implement protocols. There are modules which hold data (for
example \nedtype{RoutingTable}), facilitate communication of modules
(\nedtype{NotificationBoard}), perform autoconfiguration of a network
(\nedtype{FlatNetworkConfigurator}), move a mobile node around (for example
\nedtype{ConstSpeedMobility}), and perform housekeeping associated with
radio channels in wireless simulations (\nedtype{ChannelControl}).

Protocol headers and packet formats are described in message definition
files (msg files), which are translated into C++ classes by OMNeT++'s
\textit{opp\_msgc} tool. The generated message classes subclass from OMNeT++'s
\ttt{cPacket} or \ttt{cMessage} classes.


\section{Creating and Running Simulations}

To create a simulation, you would write a NED file that contains the network,
i.e. routers, hosts and other network devices connected together. You can
use a text editor or the IDE's graphical editor to create the network.

Modules in the network contain a lot of unassigned parameters, which need
to be assigned before the simulation can be run.\footnote{The simulator can
interactively ask for parameter values, but this is not very convenient
for repeated runs.} The name of the network to be simulated, parameter values
and other configuration option need to be specified in the \ttt{omnetpp.ini}
file.\footnote{This is the default file name; using other is also possible.}

\ttt{omnetpp.ini} contains parameter assignments as \textit{key=value}
lines, where each key is a wildcard pattern. The simulator matches these
wildcard patterns against full path of the parameter in the module tree
(something like \ttt{ThruputTest.host[2].tcp.nagleEnabled}), and value from
the first match will be assigned for the parameter. If no matching line is
found, the default value in the NED file will be used. (If there is no
default value either, the value will be interactively prompted for, or, in
case of a batch run, an error will be raised.)

OMNeT++ allows several configurations to be put into the \ttt{omnetpp.ini}
file under \ttt{[Config <name>]} section headers, and the right
configuration can be selected via command-line options when the simulation
is run. Configurations can also build on each other: \ttt{extends=<name>}
lines can be used to set up an inheritance tree among them. This feature
allows minimizing clutter in ini files by letting you factor out common
parts. (Another ways of factoring out common parts are ini file inclusion
and specifying multiple ini files to a simulation.) Settings in the
\ttt{[General]} section apply to all configurations, i.e. \ttt{[General]}
is the root of the section inheritance tree.

Parameter studies can be defined by specifying multiple values for a
parameter, with the \ttt{\$\{10,50,100..500 step 100, 1000\}} syntax;
a repeat count can also be specified.

how to run;

C++ -> dll (opp\_run) or exe

\section{Result Collection and Analysis}

how to anaylize results

how to configure result collection


\begin{note}
These are standard OMNeT++ features; please see the manual for more info.
Or go through the TicToc tutorial at least.
\end{note}

\section{Setting up wired network simulations}

todo which modules are needed into it, what they do, etc.

how to add apps, etc


\subsection{Specifying IP (IPv6) addresses in module parameters}

In INET, TCP, UDP and all application layer modules work with
both IPv4 and IPv6. Internally they use the \cppclass{IPvXAddress} C++ class, which
can represent both IPv4 and IPv6 addresses.

Most modules use the \cppclass{IPAddressResolver} C++ class to resolve addresses
specified in module parameters in omnetpp.ini.
\cppclass{IPAddressResolver} accepts the following syntax:

\begin{itemize}
  \item literal IPv4 address: \ttt{"186.54.66.2"}
  \item literal IPv6 address: \ttt{"3011:7cd6:750b:5fd6:aba3:c231:e9f9:6a43"}
  \item module name: \ttt{"server"}, \ttt{"subnet.server[3]"}
  \item interface of a host or router: \ttt{"server/eth0"}, \ttt{"subnet.server[3]/eth0"}
  \item IPv4 or IPv6 address of a host or router: \ttt{"server(ipv4)"},
      \ttt{"subnet.server[3](ipv6)"}
  \item IPv4 or IPv6 address of an interface of a host or router:
      \ttt{"server/eth0(ipv4)"}, \ttt{"subnet.server[3]/eth0(ipv6)"}
\end{itemize}


\section{Setting up wireless network simulations}

todo which modules are needed into it, what they do, etc.

\section{Setting up ad-hoc network simulations}

todo which modules are needed into it, what they do, etc.

\section{Emulation}


\section{Packet traces}

Recording packet traces

Traffic generation using packet traces

\section{Developing New Protocols}

where to put the source files: you can copy and modify the INET framework (fork it)
in the hope that you'll contribute back the changes; or you can develop in
a separate project (create new project in the IDE; mark INET as referenced project)

NED and source files in the same folder; examples under examples/; etc.

for details, read the OMNeT++ manual and the following chapters of this manual

todo...



%%% Local Variables:
%%% mode: latex
%%% TeX-master: "usman"
%%% End:

